% LaTeX Template für Abgaben an der Universität Stuttgart
% Autor: Sandro Speth
% Bei Fragen: Sandro.Speth@iste.uni-stuttgart.de
%-----------------------------------------------------------
% Hauptmodul des Templates: Hier können andere Dateien eingebunden werden
% oder Inhalte direkt rein geschrieben werden.
% Kompiliere dieses Modul um eine PDF zu erzeugen.

% Dokumentenart. Ersetze 12pt, falls die Schriftgröße anzupassen ist.
\documentclass[12pt]{scrartcl}
% Einbinden der Pakete, des Headers und der Formatierung.
% Mit den \include und \input Befehlen können Dateien eingebunden werden:
% \include: Fügt einen Seitenumbruch nach dem Text ein
% \input: Fügt KEINEN Seitenumbruch nach dem Text ein
\input{styles/Packages.tex}
\input{styles/FormatAndHeader.tex}

% Counter für das Blatt und die Aufgabennummer.
% Ersetze die Nummer des Übungsblattes und die Nummer der Aufgabe
% den Anforderungen entsprechend.
% Definiert werden die Counter in FormatAndHeader.tex
% Beachte:
% \setcounter{countername}{number}: Legt den Wert des Counters fest
% \stepcounter{countername}: Erhöht den Wert des Counters um 1.
\setcounter{sheetnr}{0} % Nummer des Übungsblattes
\setcounter{exnum}{1} % Nummer der Aufgabe

% Beginn des eigentlichen Dokuments
\begin{document}
% Nutze den \exercise{Aufgabenname} Befehl, um eine neue Aufgabe zu beginnen.
% Möchtest du eine Aufgabe in der Nummerierung überspringen, schreibe vor der Aufgabe: \stepcounter{exnum}
% Möchtest du die Nummer einer Aufgabe auf eine beliebige Zahl x setzen, schreibe vor der Aufgabe: \setcounter{exnum}{x}
\exercise{Table}
    \begin{table}[h]
        \centering
        \begin{tabular}{ c c c c }
            \hline
            \textbf{Name} & \textbf{Capital City} & \textbf{Population Density} & \textbf{Official Language(s)} \\
            \hline
            Germany & Berlin & 239 per \text{Km}{$^2$} & German \\
            The UK & London & 280 per \text{Km}{$^2$} & English \\
            The USA & Washington D.C. & 37 per \text{Km}{$^2$} & n/a \\
            China & Beijing & 152 per \text{Km}{$^2$} & Mandarin \\ 
            Japan & Tokyo & 338 per \text{Km}{$^2$} & Japanese \\
            \hline
        \end{tabular}
        \caption{Basic Statistics For Selected Countries}
        \label{countries-statistics}
    \end{table}
\exercise{}
    \begin{figure}[h]
        \centering
        \includegraphics[width=0.8\linewidth]{funny_bear}
        \caption{Description of the image.}
        \label{fig:your_image_label}
    \end{figure}


\cite{s2011}

\bibliographystyle{plain}
\bibliography{literature}


% Ende des Dokuments
\end{document}
